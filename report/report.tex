\documentclass[utf8]{article}
\usepackage{amsmath,amssymb}
\usepackage{graphicx}
\usepackage{fullpage}
\usepackage{setspace}
\usepackage{verbatim}
\usepackage{algorithm}
\usepackage{algorithmic}
\usepackage{mathrsfs}
\usepackage{listings}
\usepackage{xcolor}
\usepackage{booktabs}
\usepackage{hyperref}
\usepackage{subcaption}
\usepackage{bbm}
\hypersetup{hidelinks,
	colorlinks=true,
	allcolors=black,
	pdfstartview=Fit,
	breaklinks=true}


\lstset{
    language=bash,
    basicstyle=\ttfamily\small,
    keywordstyle=\color{blue},
    commentstyle=\color{green},
    stringstyle=\color{red},
    numbers=left,
    numberstyle=\tiny,
    stepnumber=1,
    numbersep=5pt,
    backgroundcolor=\color{gray!10},
    frame=single,
    rulecolor=\color{black!30},
    tabsize=4,
    captionpos=b,
    breaklines=true,
    breakatwhitespace=true,
    escapeinside={\%*}{*)}
}

\onehalfspacing

\title{\bf\huge Report}
\author{Minrui Luo, luomr22@mails.tsinghua.edu.cn}
\date{\today}

\begin{document}
\maketitle

\section*{Proofs}

\ \ \ \ Thm1. Angular acceleration derivation for rigid body: $\beta = I^{-1} \tau - I^{-1} [\omega] I \omega$, where $I$ is momentum of inertia according to centroid under ground reference frame, $\tau$ is total moment of force according to centroid, $\omega$ is angular velocity, $[\omega]$ is the skew matrix of $\omega$ which is defined as

\begin{equation}\label{}
  [\omega] = \begin{bmatrix}
               0 & -\omega_z & \omega_y \\
               \omega_z & 0 & -\omega_x \\
               -\omega_y & \omega_x & 0
             \end{bmatrix}
\end{equation}

(In this case: $\omega \times v = [\omega] v$)

Proof: Firstly prove some lemmas.

Lemma 1. $I_{\mathrm{centroid system}} = RI_{\mathrm{rigid body system}}R^\top$, where $R$ is the rotation matrix that transfer position vector from rigid body system to centroid system (Notice: this is a \textbf{translational system}), i.e. $r^{(\mathrm{centroid system})} = R r^{(\mathrm{rigid body system})}$.

Proof: From definition of $I$:

\begin{equation}\label{}
 \begin{aligned}
  I_{\mathrm{centroid system}} &= \int \left[\|r^{(\mathrm{centroid system})}\|^2 I_3 - r^{(\mathrm{centroid system})} {r^{(\mathrm{centroid system})}}^\top \right]\mathrm{d}m \\
  &= \int \left[\|R r^{(\mathrm{rigid body system})}\|^2 I_3 - R r^{(\mathrm{rigid body system})} (R r^{(\mathrm{rigid body system})})^\top \right]\mathrm{d}m \\
  &= \int \left[\|R r^{(\mathrm{rigid body system})}\|^2 I_3 - R r^{(\mathrm{rigid body system})} {r^{(\mathrm{rigid body system})}}^\top R^\top \right]\mathrm{d}m \\
  &= \int \left[\|r^{(\mathrm{rigid body system})}\|^2 I_3 - R r^{(\mathrm{rigid body system})} {r^{(\mathrm{rigid body system})}}^\top R^\top \right]\mathrm{d}m \\
  &= \int  \left[R \|r^{(\mathrm{rigid body system})}\|^2  I_3 R^\top - R r^{(\mathrm{rigid body system})} {r^{(\mathrm{rigid body system})}}^\top R^\top \right]\mathrm{d}m \\
  &= R \left\{ \int \left[\|r^{(\mathrm{rigid body system})}\|^2 I_3 - r^{(\mathrm{rigid body system})} {r^{(\mathrm{rigid body system})}}^\top \right]\mathrm{d}m \right\} R^\top \\
  &= R I_{\mathrm{rigid body system}} R^\top
 \end{aligned}
\end{equation}

Notice that $I$ is an (positive) symmetric matrix, so carefully choosing reference frame of rigid body system results $I_{\mathrm{rigid body system}}$ to be a \textbf{diagonal} matrix.

Lemma 2: Angular momentum theorem according to the reference point on the rigid body: $\tau = \frac{\mathrm{d} L}{\mathrm{d}t} = I_{\mathrm{centroid system}}\beta + \frac{\mathrm{d} I_{\mathrm{centroid system}}}{\mathrm{d}t} \omega$.

Proof: From definition of angular momentum, %using $v = v_{ref} + \omega \times r$,

\begin{equation}\label{}
  L = \int r \times v \mathrm{d}m %= \int r \times (v_{ref} + \omega \times r) \mathrm{d}m = \int r \times v_{ref} \mathrm{d}m + \int r \times (v_{ref} + \omega \times r) \mathrm{d}m
\end{equation}

Thus

\begin{equation}\label{}
  \frac{\mathrm{d} L}{\mathrm{d} t} = \int v \times v + r\times a \mathrm{d}m = \int r\times a \mathrm{d}m
\end{equation}

Applying Newton's second law,

\begin{equation}\label{}
  \frac{\mathrm{d} L}{\mathrm{d} t} = \int r\times \mathrm{d}F = \tau
\end{equation}

where $\tau$ is total moment of force according to this reference point.

Calculating $L$:

\begin{equation}\label{}
  L = \int r \times (\omega \times r) \mathrm{d}m = \int (r \cdot r)\omega - r(r \cdot \omega) \mathrm{d}m = \int \|r\|^2\omega - rr^\top \omega \mathrm{d}m = \left[\int \|r\|^2 - rr^\top \mathrm{d}m \right] \omega = I\omega
\end{equation}

where $I$ is the momentum of inertia under centroid (translational) reference, i.e. $I = I_{\mathrm{centroid system}}$. (Also is the momentum of inertia calculated under ground system. ) Omit subscripts where there is no ambiguity in the following text.

Without proof, using a conclusion in derivative of exponential matrix, we introduce that $\frac{\mathrm{d} R}{\mathrm{d}t} = [\omega] R$.

Lemma 3: $\frac{\mathrm{d} R^\top}{\mathrm{d}t} = -R^\top [\omega]$.

Proof: From $R R^\top \equiv I_3$, take derivative of both side:

\begin{equation}\label{}
  \frac{\mathrm{d} R}{\mathrm{d}t} R^\top + R \frac{\mathrm{d} R^\top}{\mathrm{d}t} = 0
\end{equation}

Where

\begin{equation}\label{}
  L.H.S. = [\omega] R R^\top + R \frac{\mathrm{d} R^\top}{\mathrm{d}t} = [\omega] + R \frac{\mathrm{d} R^\top}{\mathrm{d}t}
\end{equation}

Which draws the result.

Lemma 4: $\frac{\mathrm{d} I}{\mathrm{d}t} = [\omega]I - I[\omega]$.

This follows from Lemma 1 and 3 immediately.

Then the proof of Thm. 1 begins:

From Angular momentum theorem according to centroid on the rigid body, combining previous lemmas,

\begin{equation}\label{}
  \tau = \frac{\mathrm{d} L}{\mathrm{d}t} = I \beta + \frac{\mathrm{d} I}{\mathrm{d}t} \omega = I \beta + ([\omega]I - I[\omega]) \omega = I \beta + [\omega]I \omega
\end{equation}

Last step follows from $[\omega] \omega = \mathbf{0}$. Thus

\begin{equation}\label{}
  \beta = I^{-1} \tau - I^{-1}[\omega]I \omega
\end{equation}

Proof 2. Collision of rigid body

TODO.

Proof 3. Updating error of velocity, angular velocity ($O(\delta t)^3$); position, rotation matrix ($O(\delta t)^4$)

Updating rules:

\begin{equation}\label{}
  v_{t+\delta t} = v_t + \frac{3a_t - a_{t-\delta t}}{2} \delta t
\end{equation}

\begin{equation}\label{}
  \omega_{t+\delta t} = \omega_t + \frac{3\beta_t - \beta_{t-\delta t}}{2} \delta t
\end{equation}

\begin{equation}\label{}
  x_{t+\delta t} = x_t + v_t \delta t + \frac{4a_t - a_{t-\delta t}}{6} (\delta t)^2
\end{equation}

\begin{equation}\label{}
  R_{t+\delta t} = \left\{e^{[\omega(t) + \frac{4\beta(t)-\beta(t-\delta t)}{6} \delta t] \delta t} - \frac{1}{12}([\beta(t)][\omega(t)] + [\omega(t)][\beta(t)]) (\delta t)^3 \right\} R(t)
\end{equation}

Proof: For the first 2 terms,

\begin{equation}\label{}
  f(t + \delta t) = f(t) + f^\prime (t) \delta t + \frac{1}{2} f^{\prime\prime}(t) (\delta t)^2 + O((\delta t)^3)
\end{equation}

\begin{equation}\label{}
  f^\prime(t - \delta t) = f^\prime (t) - f^{\prime\prime}(t) \delta t + O((\delta t)^2)
\end{equation}

Combining those two we get

\begin{equation}\label{}
 \begin{aligned}
  f(t + \delta t) &= f(t) + f^\prime (t) \delta t + \frac{1}{2} (f^{\prime}(t) - f^\prime(t - \delta t) + O((\delta t)^2)) \delta t + O((\delta t)^3) \\
  &= f(t) + \frac{1}{2} (3f^{\prime}(t) - f^\prime(t - \delta t)) \delta t + O((\delta t)^3)
 \end{aligned}
\end{equation}

For position $x$:

\begin{equation}\label{}
  f(t + \delta t) = f(t) + f^\prime (t) \delta t + \frac{1}{2} f^{\prime\prime}(t) (\delta t)^2 + \frac{1}{6} f^{\prime\prime\prime}(t) (\delta t)^3 + O((\delta t)^4)
\end{equation}

\begin{equation}\label{}
  f^{\prime\prime}(t - \delta t) = f^{\prime\prime}(t) - f^{\prime\prime\prime}(t) \delta t + O((\delta t)^2)
\end{equation}

Combining those two

\begin{equation}\label{}
 \begin{aligned}
  f(t + \delta t) &= f(t) + f^\prime (t) \delta t + \frac{1}{2} f^{\prime\prime}(t) (\delta t)^2 + \frac{1}{6} (f^{\prime\prime}(t) - f^{\prime\prime}(t - \delta t) + O((\delta t)^2)) (\delta t)^2 + O((\delta t)^4) \\
  &= f(t) + f^\prime (t) \delta t + \frac{1}{6} (4f^{\prime\prime}(t) - f^{\prime\prime}(t - \delta t)) (\delta t)^2 + O((\delta t)^4)
 \end{aligned}
\end{equation}

For the last term (Here big-$O$ means F-norm):

\begin{equation}\label{}
  R(t+\delta t) = R(t) + \dot{R}(t) \delta t + \frac{1}{2} \ddot{R}(t) (\delta t)^2 + \frac{1}{6} \dddot{R}(t) (\delta t)^3 + O((\delta t)^4)
\end{equation}

\begin{equation}\label{}
  \dot{R}(t) = [\omega] R
\end{equation}

\begin{equation}\label{}
  \ddot{R}(t) = [\beta] R + [\omega]^2 R
\end{equation}

\begin{equation}\label{}
  \dddot{R}(t) = [\dot{\beta}] R + 2[\beta][\omega] R + [\omega][\beta] R + [\omega]^3 R
\end{equation}

Thus

\begin{equation}\label{}
 \begin{aligned}
  R(t+\delta t) &= \left\{I + [\omega(t)] \delta t + \frac{1}{2} ([\beta(t)] + [\omega(t)]^2 ) (\delta t)^2 \right. \\
  &+ \left. \frac{1}{6} ([\dot{\beta}(t)] + 2[\beta(t)][\omega(t)] + [\omega(t)][\beta(t)] + [\omega(t)]^3) (\delta t)^3 \right\} R(t) + O((\delta t)^4)
 \end{aligned}
\end{equation}

From the fact that $[a][b] = b a^\top - \langle a, b\rangle I_3$, $[a\times b] = b a^\top - a b^\top$ (actually this is from $a\times(b\times c) = b(a\cdot c) - c(a\cdot b)$, $(a\times b)\times c = b(a\cdot c) - a(b\cdot c)$)

\begin{equation}\label{}
  ([\beta][\omega] - [\omega][\beta]) = \omega \beta^\top - \beta \omega^\top = [\beta \times \omega]
\end{equation}

Combine with $\dot{\beta}(t) = \frac{\beta(t) - \beta(t-\delta t)}{\delta t} + O(\delta t)$:

\begin{equation}\label{}
 \begin{aligned}
  R(t+\delta t) &= \left\{I + [\omega(t)] \delta t + \left(\frac{1}{2} [\beta(t)] + \frac{1}{2} [\omega(t)]^2 + \frac{1}{6}[\beta(t) - \beta(t-\delta t)]\right) (\delta t)^2 \right. \\
  &+ \left. \frac{1}{6} (2[\beta(t)][\omega(t)] + [\omega(t)][\beta(t)] + [\omega(t)]^3) (\delta t)^3 \right\} R(t) + O((\delta t)^4) \\
  &= \left\{e^{[\omega(t) + \frac{4\beta(t)-\beta(t-\delta t)}{6} \delta t] \delta t} + \frac{1}{12}([\beta(t)][\omega(t)] - [\omega(t)][\beta(t)]) (\delta t)^3 \right\} R(t) + O((\delta t)^4) \\
  &= \left\{e^{[\omega(t) + \frac{4\beta(t)-\beta(t-\delta t)}{6} \delta t + \frac{1}{12}\beta(t) \times \omega(t) (\delta t)^2] \delta t} \right\} R(t) + O((\delta t)^4)
 \end{aligned}
\end{equation}

ACK: huang-wj22@mails.tsinghua.edu.cn, idea on applying Taylor's expansion of rotation matrix and checking proof details.  



\end{document} 